 
\documentclass[10pt]{article}
\special{papersize=8.5in,11.0in}
\usepackage{fullpage}

%\usepackage{amsmath}
% Packages - these are explained in the original IEEE template!
\usepackage{cite}      % Written by Donald Arseneau
\usepackage{graphicx}  % Written by David Carlisle and Sebastian Rahtz
\usepackage{url}       % Written by Donald Arseneau
\usepackage{stfloats}  % Written by Sigitas Tolusis
        
                % Gives LaTeX2e the ability to do double column
% From thesis main (bbing)
\usepackage{amssymb,longtable,dcolumn}

\begin{document}

\begin{center}
{\Large {\bf Cruise Control}} \\
{\large {Dynamics (Engr 2340), Fall 2008}} 
\end{center}

\subsection*{Instructions}
In this example we will use MATLAB to analyze one example of a first order model - the response of a car with cruise control (figure~\ref{f:msd}).

\begin{figure}[htb!]
\centerline{
{\includegraphics[width=0.6\textwidth]{cruise_control.png}}}
\caption{Cruise Control}
\label{f:msd}
\end{figure}


\begin{enumerate}
\item Write the equations of motion for the speed and forward motion of the car shown in figure~\ref{f:msd}.  Assume that the engine imparts a force $u(t)$ as shown.  Drag is modelled as a linear function of velocity.  Take the Laplace transform of the resulting differential equation and find the transfer function between the input $u$ and the output $v$.

\item Use MATLAB to find the response of the velocity of the car for the case in which the input jumps from being $u=0$ at time $t=0$ to a constant $u=500$ N thereafter.  Assuming the car's mass is 1000 kg and b = 50 Ns/m.
\item What is the steady-state speed increase for the specified input?
\item How long does it take for the car to reach this steady-state speed?
\end{enumerate}

\end{document}
